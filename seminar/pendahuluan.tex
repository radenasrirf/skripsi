%----------------------------------------------------------------------------------------
%	PENDAHULUAN
%----------------------------------------------------------------------------------------
\section*{PENDAHULUAN} % Sub Judul PENDAHULUAN
% Tuliskan isi Pendahuluan di bagian bawah ini. 
% Jika ingin menambahkan Sub-Sub Judul lainnya, silakan melihat contoh yang ada.
% Sub-sub Judul 
\subsection*{Latar Belakang}
Pengujian adalah serangkaian proses yang dirancang untuk memastikan sebuah perangkat lunak melakukan apa yang seharusnya dilakukan. Proses ini bertujuan untuk menemukan kesalahan pada perangkat lunak. Saat pengujian, bisa saja tidak ditemukan kesalahan pada Hasil pengujian. Hal ini dapat terjadi karena perangkat lunak yang sudah berkualitas tinggi atau karena proses pengujiannya berkualitas rendah. (\cite{GLENFORD2012})

Teknik pengujian secara umum dibagi menjadi 2 kategori diantaranya \textit{black box testing} dan \textit{white box testing}. \textit{Black box testing} bertujuan untuk memeriksa fungsional dari perangkat lunak apakah output sudah sesuai dengan yang ditentukan. Sedangkan \textit{White box testing} atau biasa disebut dengan pengujian struktural merupakan pemeriksaan struktur dan alur logika suatu proses. \textit{Basis Path testing} merupakan salah satu metode pengujian struktural yang menggunakan \textit{source code} dari program untuk menemukan semua jalur yang mungkin dapat dilalui program dan dapat digunakan untuk merancang data uji. Metode ini memastikan semua kemungkinan jalur dijalankan setidaknya satu kali (\cite{BASU2015}). Untuk melakukan monitoring jalur mana yang diambil oleh sebuah masukan pada saat eksekusi program, maka diperlukan penanda yang dapat memberikan informasi cabang mana yang dilalui. Proses menyisipkan tanda tersebut disebut instrumentasi. Biasanya tanda tersebut disisipkan tepat sebelum sebuah percabangan (\cite{TIKIR2011}). 

Idealnya, pengujian dilakukan untuk semua kemungkinan dari perangkat lunak. Tetapi untuk menguji perangkat lunak yang kompleks secara keseluruhan akan memakan waktu yang lama dan membutuhkan sumber daya manusia yang banyak. \citeauthor{KUMAR20168} (\cite*{KUMAR20168}) mengatakan bahwa pengujian perangkat lunak menggunakan hampir 60\% dari total biaya pengembangan perangkat lunak. Jika proses pengujian perangkat lunak dapat dilakukan secara otomatis, maka hal ini dapat mengurangi biaya pengembangan secara signifikan. 

\citeauthor{HERMADI2015} (\cite*{HERMADI2015}) melakukan penelitian untuk membangkitkan data uji untuk \textit{path testing} menggunakan algoritma genetika. Dalam penelitian tersebut, \citeauthor{HERMADI2015} membangkitkan \textit{Control flow Graph} (CFG) dan instrumentasi masih secara manual sehingga membutuhkan banyak waktu dan rawan akan kesalahan ketika program sudah semakin besar. Sehingga mengotomasi hal tersebut dapat membuat \textit{path testing} menjadi lebih cepat dan mengurangi kerawanan akan kesalahan.

Pada penelitian ini, akan dibangun sebuah perangkat lunak untuk membangkitkan kemungkinan jalur dari sebuah program. Jalur-jalur ini dapat dijadikan dasar untuk membangkitkan data uji agar data uji yang digunakan untuk pengujian dapat mewakili semua kemungkinan. Untuk memonitor jalur mana yang dilalui ketika diberikan masukan data uji, maka sistem ini juga akan melakukan penyisipan tag-tag sebagai instrumentasi ke dalam \textit{source code} secara otomatis. 

% Sub-sub Judul 
\subsection*{Perumusan Masalah}
Berdasarkan latar belakang di atas dapat dirumuskan masalahnya adalah bagaimana membangun sebuah aplikasi untuk melakukan instrumentasi secara otomatis untuk pengujian jalur dan \textit{re-engineering} perangkat lunak.

\subsection*{Tujuan}
Penelitian ini bertujuan untuk membangun sebuah aplikasi yang dapat digunakan untuk membangkitkan CFG dan melakukan instrumentasi secara otomatis.

\subsection*{Ruang Lingkup}
Bahasa pemrograman yang diakomodasi adalah Matlab dan model diagram yang dibangkitkan adalah CFG.

\subsection*{Manfaat}
Hasil penelitian diharapkan dapat membantu pengembang dan penguji aplikasi untuk:
\begin{enumerate}[noitemsep] 
\item Menyisipkan tag-tag sebagai instrumentasi program ke dalam \textit{source code} secara otomatis sehingga proses tersebut dapat dilakukan dengan lebih cepat.
\item Membangkitkan jalur-jalur dasar yang dapat digunakan sebagai dasar untuk pembangkitan data uji.
\item Membangkitkan diagram CFG yang dapat memudahkan pengembang dalam memahami struktur dan alur dari suatu program yang dapat dimanfaatkan ketika akan melakukan \textit{re-engineering} perangkat lunak.
\end{enumerate}
