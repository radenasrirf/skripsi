%----------------------------------------------------------------------------------------
%	PENDAHULUAN
%----------------------------------------------------------------------------------------

\section*{PENDAHULUAN} % Sub Judul PENDAHULUAN
% Tuliskan isi Pendahuluan di bagian bawah ini. 
% Jika ingin menambahkan Sub-Sub Judul lainnya, silakan melihat contoh yang ada.
% Sub-sub Judul 

\subsection*{Latar Belakang}
Subsektor peternakan memiliki peranan penting dalam perekonomian Indonesia baik dalam pembentukan Produk Domestik Bruto (PDB) dan penyerapan tenaga kerja maupun dalam penyediaan bahan baku industri. Perannya dalam pertumbuhan ekonomi menunjukkan bahwa PDB peternakan triwulan I tahun 2005 tumbuh 5.8\%. Kontribusi PDB subsektor peternakan terhadap sektor pertanian triwulan I tahun 2005 mencapai 13.2\%. Sedangkan terhadap besaran PDB Nasional mencapai 2\%. Dalam penyerapan tenaga kerja sub sektor peternakan juga mempunyai peranan yang sangat strategis. Menurut hasil sensus pertanian 2003 dari 24,86 juta Rumah Tangga Pertanian di pedesaan dan perkotaan, sekitar 22,63\% merupakan Rumah Tangga Usaha Peternakan. Selain itu sub sektor peternakan juga berperan penting dalam penyediaan bahan baku bagi keperluan industri (\cite{Makka2012}). 

Efisiensi produksi dalam suatu usaha peternakan menjadi faktor penentu keberhasilan peternakan. Efisiensi produksi dapat diwujudkan dengan pemberian pakan yang berkualitas dengan kuantitas yang memadai sesuai dengan kebutuhan ternak. Pakan merupakan salah satu aspek yang sangat penting dalam keberhasilan suatu usaha peternakan. Sehingga formulasi ransum dari sejumlah bahan pakan yang tersedia merupakan aspek yang sangat vital khususnya dalam rangka menyeimbangkan kandungan energi, protein dan nutrien lainnya (\cite{Jayanegara2014}). Berdasarkan sudut pandang ekonomi, biaya untuk pembelian pakan ternak merupakan biaya tertinggi dalam agribisnis perternakan. Sehingga biaya tersebut harus ditekan serendah mungkin agar tidak mengurangi pendapatan. Teknologi dapat menjadi jalan keluar dalam permasalahan tersebut, yaitu dengan mengaplikasikan teknologi formulasi pakan ternak yang efisien. Pakan ternak yang diramu dengan baik dan sesuai dengan kebutuhan ternak akan menekan biaya pembelian pakan serendah mungkin (\cite{Shiddieqy2010}).

Ransum yang murah dan berkualitas memerlukan suatu teknik atau metode dalam memformulasikannya. Formulasi ransum yang mudah digunakan, cepat, akurat dalam penentuan komposisi bahan dan mendapatkan biaya serendah mungkin dapat menggunakan metode \textit{linier programming}. Selain metode \textit{linier programming}, ada beberapa metode lain yang dapat digunakan, antara lain metode \textit{trial and error}, \textit{equation} dan \textit{pearson’s square}. Diantara metode-metode tersebut, metode \textit{linier programming} adalah yang paling sesuai untuk diterapkan sebagai metode formulasi ransum karena mampu menangani jumlah variabel yang banyak secara efisien (\cite{Muzayyanah2013}). Akan tetapi dalam penghitungan secara manual metode ini masih dirasa sangat sulit (\cite{Kusnandar2004}).

Penelitian tentang formulasi ransum ternak sapi potong sudah pernah dilakukan oleh \citeauthor{Rahman2017} (\cite*{Rahman2017}). Peneliti mengembangkan sistem formulasi ransum ternak sapi potong berdasarkan nilai ADG (\textit{average daily gain}) dan berat badan ternak menggunakan metode linier programming. Sistem tersebut dapat melakukan formulasi dengan kesamaan dan akurasi yang baik karna hasil perbandingan mendapatkan selisih 0. Namun pada penelitian sebelumnya terdapat beberapa kekurangan. Sehingga pada penelitian ini diusulkan sebuah sistem pengembangan dari sistem informasi yang telah dikembangkan oleh \citeauthor{Rahman2017} (\cite*{Rahman2017}) dengan hewan ternak mencakup seluruh ternak ruminansia.

% Sub-sub Judul 
\subsection*{Perumusan Masalah}
Berdasarkan uraian yang tercantum pada latar belakang maka perumusan masalah pada penelitian ini adalah :
\begin{enumerate}[noitemsep] 
	\item Bagaimana menerapkan algoritme \textit{linier programming }untuk formulasi ransum ternak ruminansia?
	\item Bagaimana hasil evaluasi ransum pada formulasi?
\end{enumerate}

\subsection*{Tujuan}
Tujuan dari penelitian ini adalah:
\begin{enumerate}[noitemsep]
	\item Mengembangkan sistem formulasi ransum menggunakan algoritme linier programming untuk ternak ruminansia.
	\item Melakukan evaluasi ransum hasil formulasi sistem dengan hasil formulasi pakar.
\end{enumerate}

\subsection*{Ruang Lingkup}
Lingkup dari penelitian ini adalah pengembangan sistem berfokus pada penghitungan formulasi ransum.

\subsection*{Manfaat}
Hasil penelitian diharapkn dapat membantu para peternak dalam melakukan formulasi ransum secara cepat dan tepat.