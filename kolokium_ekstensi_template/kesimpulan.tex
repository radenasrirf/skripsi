%----------------------------------------------------------------------------------------
%	KESIMPULAN DAN SARAN
%----------------------------------------------------------------------------------------
\section*{KESIMPULAN DAN SARAN}
\subsection*{Kesimpulan}
Penelitian yang dilakukan telah berhasil mengembangkan formulasi ransum yang mampu mengatur batasan minimum dan maksimum pakan yang digunakan serta nutrisi yang dibutuhkan dengan mengoptimalkan harga ransum menggunakan metode linier programming. Metode pengembangan sistem pada penelitian menggunakan metode prototyping dan memiliki 2 iterasi. Hasil akhir penelitian menunjukkan bahwa sistem formulasi ini dapat menghasilkan harga dengan nilai akurasi 0.81\% jika dibandingkan dengan aplikasi WinFeed.

\subsection*{Saran}
Penelitian ini memiliki beberapa kekurangan yang dapat dikembangkan pada penelitian selanjutnya. Penelitian selanjutnya dapat memperbaiki hasil formulasi jika terjadi \textit{infeasible }untuk dianalisis dan ditampilkan variabel yang terlalu dekat dan menyebabkan hasil \textit{infeasible}. Penelitian selanjutnya juga dapat mengembangkan pada bagian pemilihan pakan untuk ditampilkan apakah pakan yang dipilih sudah memenuhi nutrien yang dibutuhkan sebelum dilakukan formulasi.