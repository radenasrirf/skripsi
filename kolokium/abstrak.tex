%----------------------------------------------------------------------------------------
%	ABSTRACT
%----------------------------------------------------------------------------------------
\Abstract{\scriptsize 
% ---- Tuliskan abstrak di bagian ini seperti contoh.
Pengujian adalah serangkaian proses yang dirancang untuk memastikan sebuah perangkat lunak melakukan apa yang seharusnya dilakukan dan bertujuan untuk menemukan kesalahan pada perangkat lunak. \textit{Basis Path testing} merupakan salah satu metode pengujian struktural yang menggunakan \textit{source code} dari program untuk menemukan semua jalur yang mungkin dapat dilalui program dan dapat digunakan untuk merancang data uji. Untuk menguji perangkat lunak yang kompleks secara keseluruhan akan memakan waktu yang lama dan membutuhkan sumber daya manusia yang banyak. Idealnya, pengujian dilakukan untuk semua kemungkinan dari perangkat lunak. \citeauthor{KUMAR20168} (\cite*{KUMAR20168}) mengatakan bahwa pengujian perangkat lunak menggunakan hampir 60\% dari total biaya pengembangan perangkat lunak. Sehingga mengotomasi bagian dari pengujian akan membuat proses ini menjadi lebih cepat dan mengurangi kerawanan akan kesalahan. Pada penelitian ini, akan dibangun sebuah sistem untuk membangkitkan kemungkinan jalur-jalur dari sebuah program yang dapat dijadikan dasar untuk membangkitkan data uji agar data uji yang digunakan untuk pengujian dapat mewakili semua kemungkinan. Untuk memonitor jalur yang dilalui program ketika dijalankan dengan masukan data uji tertentu, maka sistem ini juga akan melakukan instrumentasi \textit{source code} program secara otomatis. Program yang akan diuji dalam penelitian ini adalah program yang dibangun dengan menggunakan bahasa Matlab. Dalam pengembangannya, aplikasi ini akan dibangun dengan menggunakan bahasa pemrograman Java dan library \textit{Graphviz} untuk memvisualisasikan \textit{Control Flow Graph}.

% ---- Akhir bagian abstrak
\normalsize}
